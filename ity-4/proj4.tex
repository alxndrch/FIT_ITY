\documentclass[11pt, a4paper]{article}

\usepackage[left=2cm,text={17cm, 24cm},top=3cm]{geometry}
\usepackage[czech]{babel}
\usepackage[IL2]{fontenc}
\usepackage[utf8]{inputenc}
\usepackage{times}
\usepackage{graphicx}
\usepackage{amsmath,amsthm,amsfonts}
\usepackage{array}
\usepackage[procnames]{listings}
\usepackage{color}
\usepackage{url}

%https://tex.stackexchange.com/questions/12703/how-to-create-fixed-width-table-columns-with-text-raggedright-centered-raggedlef
\newcolumntype{L}[1]{>{\raggedright\let\newline\\\arraybackslash\hspace{0pt}}m{#1}}
\newcolumntype{C}[1]{>{\centering\let\newline\\\arraybackslash\hspace{0pt}}m{#1}}
\newcolumntype{R}[1]{>{\raggedleft\let\newline\\\arraybackslash\hspace{0pt}}m{#1}}


\begin{document}
\begin{titlepage}
    \begin{center}
        \Huge{\textsc{Vysoké učení technické v~Brně}}\\
        \huge{\textsc{Fakulta informačních technologií}}\\
        \vspace{\stretch{0.382}}
        \LARGE{Typografie a publikování -- 4. projekt}\\
        \Huge{Bibliografické citace}\\
        \vspace{\stretch{0.618}}
    \end{center}
    {\Large \today \hfill Alexandr Chalupnik (xchalu15)}
\end{titlepage}

\section{Tvorba dokumentů v sysému \LaTeX{}}
\subsection{Co je to \LaTeX{}}
Za LaTeX lze v širší rovině považovat typografický systém, který je primárně určen k sazbě vědeckých a matematických dokumentů vysoké typografické kvality. Tento systém je každopádně vhodný i pro tvorbu rozličných dokumentů jako jsou například jednoduché vizitky nebo několikasetstránkové knihy.\cite{web:medium}

\subsection{Historie}
Předchůdcem \LaTeX{}u od Leslie Lamporta byl sázecí systém \TeX, který vytvořit profesor Donald Knuth kolem roku 1983. \TeX umožňoval svým uživatelům jednoduchou sazbu textu a zejména matematických rovnic. \cite{thesis:history} 

\subsection{Sázení tabulek}
Pokud chceme vysázet tabulku, můžeme použít dvě prostředí. Jedním z nich je \verb|tabbing|, které lze přirovnat k tabelačním zarážkám, které tvoří tabulku tím že zarovnávají sloupce. Druhým prostředím je \verb|tabular|, které slouží k tvorbě složitějších tabulek. \cite{web:table}

\renewcommand{\arraystretch}{1.5}
\begin{table}[h!] \catcode`\-=12
\setlength\arrayrulewidth{1pt}
\small
\centering
    \begin{tabular}{|C{3.5cm}|C{2cm}|C{3.5cm}|}\hline
         Kernel function &
         Accuracy &
         Parameter \\ \hline
         Linear & 88\% (22/25) & $c =$ 2, $g =$ 1\\ \hline
         Polynomial & 92\% (23/25) & $c =$ 2, $g =$ 1\\ \hline
         \textbf{Radial basis function} & \textbf{96\% (24/25)} & \textbf{$c =$ 2, $g =$ 1}\\ \hline
         Sigmoid & 68\% (17/25) & $c =$ 2, $g =$ 1\\ \hline
    \end{tabular}
    \caption{Příklad tabulky vytvořené v prostředí \texttt{tabular} \cite{article:table}} 
\end{table}
\renewcommand{\arraystretch}{1}


\subsection{Sázení vzorců}
Jedním z důvodů proč je \LaTeX{} oblíbený, je právě možnost velmi propracovaného a jednoduchého sázení matematických výraz. Pro sázení existuje několik prostředí.\cite{book:rybicka}
V případě, že se vzorec zapisuje uvnitř odstavce, používá se pro oddělení matematického prostředí znak \verb|$|. Pokud má být vzorec sázen na samostatný řádek, odděluje se symboly \verb|\[| \dots \verb|\]| nebo \verb|$$| \dots \verb|$$|. Dalším možným prostředím je \verb|eqnarray|, to je určeno pro sazbu posloupnosti vzorců pod sebe.\cite{web:math} Příklad vysázení rovnice na samostatný řádek\cite{article:math}:

$$
a \oplus b = 
(a + b)\left(1+ \frac{\bar{a}b}{t^2}\right)^{-1} = 
\left(1+ \frac{b\bar{a}}{t^2}\right)^{-1} (a+b)
$$


\subsection{Vkládání map}
\LaTeX{} dokonce umožňuje vkládání map do dokumentu. Slouží k tomu balíček \verb|getmap|, který získává mapy z externích zdrojů jako jsou OpenStreetMap nebo Google Maps včetně podpory Google StreetVeiw. 
Pro získání mapy slouží příkaz ve tvaru \verb|\getmap[volby]{adresa}|. Získanou mapu vložíme do dokumentu pomocí příkazu \verb|\includegraphics|, který nalezneme v balíčku \verb|graphicx|. Pro správnou činnosti příkazu \verb|getmap| je potřebné dokument kompilovat s přepínačme \verb|--shell-escape|.\cite{article:map}

\newpage
\subsection{Vkládání algoritmů}
Pro sázení algoritmů a ukázek kódů v některém programovacím jazyce, je možné využít jeden ze tří balíčků: \verb|argorthm|, \verb|algorithm2e| a \verb|listings|.
Ukázka kódu v jazyce Python pomocí balíčku \verb|listings| \cite{book:python}:

%https://python.g-node.org/python-summerschool-2009/python_code_in_latex.html
\definecolor{keywords}{RGB}{255,0,90}
\definecolor{comments}{RGB}{0,0,113}
\definecolor{red}{RGB}{160,0,0}
\definecolor{green}{RGB}{0,150,0}
 
\lstset{language=Python, 
        basicstyle=\ttfamily\small, 
        keywordstyle=\color{keywords},
        stringstyle=\color{red},
        identifierstyle=\color{green},
        procnamekeys={def,class}}

\begin{lstlisting}[language=Python]
inp=input('Enter Fahrenheit Temperature:')
try:
    fahr=float(inp)
    cel=(fahr-32.0)*5.0/9.0
    print(cel)
except:
    print('Please enter a number')
\end{lstlisting}

Python je populární programovací jazyk, který má mnoho využití. Lze ho použít pro tvorbu webových aplikací, pro vizualizaci dat a numerické výpočty. Díky sve jednoduchosti je často využíván při výuce programování.\cite{thesis:python}

\bibliographystyle{czechiso}
\renewcommand{\refname}{Literatura}
\bibliography{proj4}

\end{document}
