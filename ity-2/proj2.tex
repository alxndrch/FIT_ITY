\documentclass[11pt, a4paper,twocolumn]{article}

\usepackage[left=1.5cm,text={18cm, 25cm},top=2.5cm]{geometry}
\usepackage[czech]{babel}
\usepackage[IL2]{fontenc}
\usepackage{times}
\usepackage{amsmath,amsthm,amsfonts}

\def \fnote {Pro libovolnou abecedu $\Sigma$ je $\Sigma^\omega$ množina všech \emph{nekonečných} řetězců nad $\Sigma$, tj. nekonečných posloupností symbolů ze $\Sigma$.}

\newtheorem{theorem}{Věta}
\newtheorem{definition}{Definice}
 
\begin{document}
\begin{titlepage}
    \begin{center}
        \Huge
        \textsc{Fakulta informačních technologií}\\
        \textsc{Vysoké učení technické v Brně}\\
        \vspace{\stretch{0.382}}
        \LARGE{Typografie a publikování -- 2. projekt\\[0.4em]
        Sazba dokumentů a matematických výrazů}\\[0.3em]
        \vspace{\stretch{0.618}}
    \end{center}
    {\Large 2020 \hfill Alexandr Chalupnik (xchalu15)}
\end{titlepage}

\section*{Úvod} 
V~této úloze si vyzkoušíme sazbu titulní strany, matematických vzorců, prostředí a dalších textových struktur obvyklých pro technicky zaměřené texty (například rovnice (\ref{eq:eq2}) nebo Definice \ref{def:def2} na straně \pageref{eq:eq1}). Pro vytvoření těchto odkazů používáme příkazy \verb|\label|, \verb|\ref| a \verb|\pageref|. 

Na titulní straně je využito sázení nadpisu podle optického středu s~využitím zlatého řezu. Tento postup byl probírán na přednášce. Dále je použito odřádkování se zadanou relativní velikostí 0{,}4\,em a 0{,}3\,em.

\section{Matematický text}
Nejprve se podíváme na sázení matematických symbolů a~výrazů v~plynulém textu včetně sazby definic a vět s~využitím balíku \verb|amsthm|. Rovněž použijeme poznámku pod čarou s~použitím příkazu \verb|\footnote|. Někdy je vhodné použít konstrukci \verb|${}$| nebo \verb|\mbox{}| která říká, že (matematický) text nemá být zalomen. V~následující definici je nastavena mezera mezi jednotlivými položkami \verb|\item| na 0,05\,em.

\begin{definition} \label{def:def1}
\textnormal{Turingův stroj} (TS) je definován jako šestice tvaru $M = (Q, \Sigma, \Gamma, \delta, q_0, q_F)$, kde:
\begin{itemize}
  \setlength\itemsep{0.05em}
  \item $Q$ je konečná množin \textnormal{vnitřních (řídicích) stavů},
  \item $\Sigma$ je konečná množina symbolů nazývaná \textnormal{vstupní abeceda}, $\Delta \notin \Sigma $, 
  \item $\Gamma$ je konečná množina symbolů $\Sigma \subset \Gamma$, $\Delta \in \Gamma$, nazývaná \textnormal{pásková abeceda},
  \item $\delta$ : $(Q \setminus \{q_F\}) \times \Gamma \to Q \times (\Gamma \cup \{L,R\})$, kde $L,R \notin \Gamma$, je parciální \textnormal{přechodová funkce}, a
  \item $q_0\in Q$ je \textnormal{počáteční stav} a $q_f \in Q$ je \textnormal{koncový stav}.
\end{itemize}
\end{definition}

Symbol $\Delta$ značí tzv. \emph{blank} (prázdný symbol), který se vyskytuje na místech pásky, která nebyla ještě použita.

\emph{Konfigurace pásky} se skládá z~nekonečného řetězce, který reprezentuje obsah pásky a pozice hlavy na tomto řetězci. Jedná se o~prvek množiny ${\{ \gamma \Delta^\omega \mid \gamma \in \Gamma^\ast \} \times \mathbb{N}}$\footnote{\fnote}.
\emph{Konfiguraci pásky} obvykle zapisujeme jako ${\Delta xyz \underline{z}x \Delta\ldots}$ (podtržení značí pozici hlavy). \emph{Konfigurace stroje} je pak dána stavem řízení a konfigurací pásky. Formálně se jedná o~prvek množiny ${Q \times \{\gamma \Delta^\omega \mid \gamma \in \Gamma^\ast \} \times \mathbb{N}}$ 

\subsection{Podsekce obsahující větu a odkaz}
\begin{definition} \label{def:def2}
\textnormal{Řetězec $w$ nad abecedou $\Sigma$ je přijat TS $M$} jestliže $M$ při aktivaci z~počáteční konfigurace pásky
$\underline{\Delta}w\Delta\ldots$ a počátečního stavu $q_0$ zastaví přechodem do koncového stavu $q_F$ tj. $(q_0,\Delta w\Delta^\omega,0)  \vdash_M^* (q_F,\gamma,n)$ pro nějaké $\gamma \in \Gamma^\ast  $ a $n \in \mathbb{N}$.

Množinu $L(M) = \{w\!\mid\! w$ je přijat TS $M\}  \subseteq \Sigma^\ast $ na\-zý\-vá\-me jazyk přijímaný \textnormal{TS} $M$.
\end{definition}

Nyní si vyzkoušíme sazbu vět a důkazů opět s~použitím balíku \texttt{amsthm}.

\begin{theorem}
Třída jazyků, které jsou přijímány TS, odpovídá \textnormal{rekurzivně vyčíslitelným jazykům.}
\end{theorem}

\begin{proof}
V~důkaze vyjdeme z~Definice \ref{def:def1} a \ref{def:def2}.
\end{proof}

\section{Rovnice}
Složitější matematické formulace sázíme mimo plynulý text. Lze umístit několik výrazů na jeden řádek, ale pak je třeba tyto vhodně oddělit, například příkazem \verb|\quad|.

$$
\sqrt[i]{x_i^{3}}
   \quad \text{kde $x_i$ je $i$-té sudé číslo} \quad 
y_i^{2\cdot y_i} \neq y_i^{y_i^{y_i}}
$$

V~rovnici (\ref{eq:eq1}) jsou využity tři typy závorek s~různou explicitně definovanou velikostí.

\begin{eqnarray} 
x & = & \Bigg\{ \bigg( \big[a+b\big]\ast c \bigg)^d \oplus 1 \Bigg\} \label{eq:eq1} \\ 
y & = & \lim_{x \to \infty}\frac{\sin^2 x + \cos^2 x}{\frac{1}{\log_{10} x}} \label{eq:eq2}
\end{eqnarray}

V~této větě vidíme, jak vypadá implicitní vysázení limity $lim_{n \to \infty} f(n)$ v~normálním odstavci textu. Podobně je to i s~dalšími symboly jako $\sum_{i=1}^n 2^i$ či $\bigcap_{A \in \mathcal{B}} A$. V~případě vzorců $\lim\limits_{x \to \infty} f(n)$ a $\sum\limits_{i=1}^{n}2^i$ jsme si vynutili méně úspornou sazbu příkazem\verb|\limits|.

\section{Matice}
Pro sázení matic se velmi často používá prostředí array a~závorky (\verb|\left|,\verb|\right|).

$$
\left( 
\begin{array}{ccc} 
    a+b & \widehat{\xi+\omega} & \hat{\pi} \\
    \vec{\mathbf{a}} & \overleftrightarrow{AC} & \beta \\
\end{array} \right)
= 1 \Longleftrightarrow \mathbb{Q} = \mathcal{R}
$$
Prostředí array lze úspěšně využít i jinde.

$$
\binom{n}{k} = \left\{
\begin{array}{c l}
    0 & \text{pro } k < 0 \text{ nebo } k >n \\
    \frac{n!}{k!(n-k)!} & \text{pro } 0 \leq k \leq n.
\end{array}
\right.
$$

\end{document}